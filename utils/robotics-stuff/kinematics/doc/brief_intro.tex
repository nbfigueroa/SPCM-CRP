\documentclass[a4paper,10pt]{article}


%opening
\title{}
\author{}

\begin{document}

The special Euclidean group (3) is commonly know in the robotics literature as
homogeneous transformations.  The Lie algebra of SE(3), denoted $se(3)$, is
identified by a $4\times4$ skew symmetric matrix of the form:
\begin{equation}
\left[\begin{array}{c c c c}
  0 & -\omega_3 & \omega_2 & v_1 \\
  \omega_3 & 0 & -\omega_1 & v_2 \\
  -\omega_2 & \omega_1 & 0 & v_3 \\
  0 & 0 & 0 & 0 \\
\end{array}\right] = \hat{\xi}
\end{equation}
The mapping from $se(3)$ to $SE(3)$ is performed
by the exponential formula $H = e^{\hat{\xi}}$ and a closed-form solution
exists through the Rodriguez formula.  We refer to the matrix $\hat{\xi}$ as a
{\it twist}. Similar to Murray, we define the $\vee$
(vee) operator to extract the six-dimensional {\it twist coordinates}
which parametrize a twist,
\begin{equation}
\left[\begin{array}{c c c c}
  0 & -\omega_3 & \omega_2 & v_1 \\
  \omega_3 & 0 & -\omega_1 & v_2 \\
  -\omega_2 & \omega_1 & 0 & v_3 \\
  0 & 0 & 0 & 0 \\
\end{array}\right]^{\vee} =
\left[\begin{array}{c}
  v_1 \\ v_2 \\ v_3 \\ \omega_1 \\ \omega_2 \\ \omega_3 \\
\end{array}\right] = \xi
\end{equation}
The motion between consecutive frames can be represented by right
multiplication of $H$ with a motion matrix $M$.  

The adjoint operator provides a convenient
method for transforming a twist from one coordinate frame to
another. Given $M \in SE(3)$, the adjoint transform is a $6
\times 6$ matrix which transforms twists from one coordinate frame to another.

\begin{equation}
  M =
  \left[\begin{array}{c c}
    R & {\bf t} \\
    {\bf 0_{1 \times 3}} & 1 \\
  \end{array}\right]
\end{equation}

\begin{equation}
  Ad(M) =
  \left[\begin{array}{c c}
    R & {\bf \hat{t}} R \\
    {\bf 0_{3 \times 3}} & R
  \end{array}\right]
\end{equation}

\noindent The adjoint operator is invertible, and is given by:

\begin{equation}
  Ad^{-1}(M) =
  \left[\begin{array}{c c}
    R^T & -R^T \hat{\bf t} \\
    {\bf 0_{3 \times 3}} & R^T
  \end{array}\right]
\end{equation}

\end{document}